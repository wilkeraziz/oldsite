
\section*{INVITED TALKS}

\begin{tabular}{p{2cm}  p{13.92cm}}
04/2014 & Invited talk, University of Amsterdam, Amsterdam, The Netherlands \\
        & {\bf Title:} Exact Inference for Statistical Machine Translation \\
        & {\bf Summary:} In this presentation I talk about exact decoding and unbiased sampling for hierarchical and phrase-based SMT based on a coarse-to-fine strategy. In this view the intractable intersection between the translation forest and the language model is replaced by a simpler, thus tractable, intersection with a lower-order upperbound on the true LM distribution. The resulting distribution is then incrementally refined in an adaptive rejection sampling fashion.  \\
%        OS* is a tractable form of adaptive rejection sampling that can also be used for optimisation. The contributions of this research go beyond the exactness aspect of OS*. Sampling has many applications in SMT, such as it enables one to better explore the space of likely solutions, it is less prone to outliers than optimisation, it is relevant to topics such as minimum error rate training, minimum Bayes risk decoding, and consensus decoding. Topics relevant to this talk are: SMT, SMT decoding, automata theory, complexity theory, optimisation and sampling. 
        & \\
05/2013 & Invited talk, University of Sheffield, Sheffield, UK \\
        & {\bf Title:} Exact Optimisation and Sampling for Statistical Machine Translation \\
        & {\bf Summary:} Preliminary findings of my PhD where I present the OS* algorithm (Dymetman et al, 2012) and how this algorithm can be used to perform exact optimisation and sampling for SMT. \\
        & \\
09/2011	& Tutorial, RANLP, Hissar, Bulgaria \\
	& {\bf Summary:} Together with Lucia Specia I gave a 3-hour tutorial on SMT. \\
	& \\
10/2011	& Invited Talk, Universidade de S\~ao Paulo, S\~ao Carlos, Brazil \\
	& {\bf Title:} Improving Chunk-based Semantic Role Labelling with Lexical Features \\
	& {\bf Summary:} Findings of my investigation on improving semantic role labelling by using lexical 	features published at RANLP-2011.  \\
	& \\
01/2010	& Invited talk, University of Wolverhampton, Wolverhampton, UK \\
	& {\bf Title:} Learning an Expert from Human Annotations in Statistical Machine Translation: the 	Case of Out-of-Vocabulary Words \\
	& {\bf Summary:} Findings of the work I developed at XRCE on handling unknown words 	using Textual Entailment and incorporating an expert model into a standard SMT system. 
	%Out-of-vocabulary words are replaced at decoding time by known synonyms 	and hypernyms then these alternatives are translated. Replacement candidates are ranked using  	different context models added as features in a log-linear model. I proposed a learning scheme to tune the additional components having fixed the vector of weights of the original SMT system, in this way my model can be seen as an expert working on the top of a well established baseline model, affecting its original performance only in the presence of out-of-vocabulary words. 
	\\
\end{tabular}

%Additional lexical evidence is given to a standard CRF-based semantic role labeller in the form of: i) automatically extracted selectional preferences (linguistically motivated), ii) automatically inferred similar words (corpus-based). Selectional preferences are automatically annotated using WordNet as the structured knowledge base and similar words are queried from Lin's thesaurus. The additional lexical information led to considerable gains in recall for chunk-based SRL in the absence of deep syntactic features.
