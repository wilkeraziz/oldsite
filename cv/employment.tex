\section*{Employment}

\begin{tabular}{p{2.92cm}  p{13cm}}
01/2019--present & \bf Assistant Professor, Institute for Logic, Language and Computation, Universiteit van Amsterdam, Netherlands \\
	& {\bf Summary:} My research group focuses on probabilistic models of language with a strong focus on unsupervised learning and latent variable modelling. \\
	& \\	
01/2015--12/2018 & \bf Research Associate, Institute for Logic, Language and Computation, Universiteit van Amsterdam, Netherlands \\
	& {\bf Summary:} I joined the Statistical Language Processing and Learning Lab led by Professor Khalil Sima'an in January 2015 where I worked on several aspects of machine translation (e.g. word alignment, word reordering, and morphological analysis and generation) and paraphrasing employing log-linear, Bayesian, and deep generative models.
	 \\
    & \\
11/2013--12/2014 & \bf Research Associate, Department of Computer Science, University of Sheffield, UK \\
	& {\bf Summary:} My work was funded by EPSRC under the MODIST (MOdelling DIscourse in Statistical Translation) project led by Prof. Dr. Lucia Specia. Discourse information typically requires nonlocal forms of parameterisation. % that go beyond the narrow context window managed by traditional decoders. 
	I developed better decoding algorithms for SMT aiming at incorporating global features, particularly, I worked on a lazy incorporation of nonlocal parameterisation using a form of adaptive rejection sampling. 
	 \\
    & \\
08/2013--12/2013	& \bf Internship, Xerox Research Centre Europe (XRCE), Grenoble, France \\
	& {\bf Summary:} I worked with the Machine Learning for Document Access and Translation group under supervision of Dr. Marc Dymetman and Dr. Sriram Venkatapathy  on developing an exact decoder/sampler for phrase-based SMT.  \\%Exact inference is achieved with the OS* algorithm (a technique previously developed at Xerox), a form of adaptive rejection sampling that can also be used for optimisation. \\
	& \\
%07/2010--10/2010 & \bf Junior Engineer, Neolog, S\~ao Paulo, Brazil \\
%	& {\bf Summary:} I worked on the composition of classic optimisation problems. The company develops solutions for optimising supply chains. During my short stay I worked on the refactoring of core components, such as the search itself. The search algorithms were somewhat exhaustive, therefore inefficient for large scale applications. I worked on a framework of tabu search bringing techniques such as beam-search to it. \\
%	& \\
%03/2010--06/2010	& \bf Tutoring, Instituto de Ci\^encias Matem\'aticas e de Computa\c{c}\~ao - USP \\
%	& {\bf Summary:} I gave weekly tutoring sessions on Automata Theory, Formal Language and Theory of Computation to computer science undergraduates. \\
%	& \\
03/2009--02/2010	& \bf Internship, Xerox Research Centre Europe (XRCE), Grenoble, France \\
	& {\bf Summary:} I worked with the Cross-Language Technologies group under supervision of Dr. Marc Dymetman and Dr. Lucia Specia on the use of context models and textual entailment to improve statistical machine translation coverage and quality. \\ %My project on ``Context Models for Textual Entailment and their Application to Statistical Machine Translation'' was part of a project funded by Pascal-2 European Network of Excellence  and it was granted the first prize of the September 2009's XRCE Intern's Day. \\
	& \\
%07/2006--12/2006	& \bf Tutoring, Instituto de Ci\^encias Matem\'aticas e de Computa\c{c}\~ao - USP \\
%	& {\bf Summary:} I gave weekly tutoring sessions on Linear Algebra and Ordinary Differential Equations for computer science and chemistry undergraduates. \\
%	& \\
%05/2003--04/2004	& \bf	Internship, Fun\c{c}\~ao Inform\'atica LTDA - S\~ao Paulo, Brazil  \\
%	& {\bf Summary:} It was a technical level internship in which I developed general purpose programming, network and database management skills. \\
\end{tabular}
